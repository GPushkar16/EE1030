%iffalse
\let\negmedspace\undefined
\let\negthickspace\undefined
\documentclass[journal,12pt,onecolumn]{IEEEtran}
\usepackage{cite}
\usepackage{amsmath,amssymb,amsfonts,amsthm}
\usepackage{algorithmic}
\usepackage{multicol}
\usepackage{graphicx}
\usepackage{textcomp}
\usepackage{xcolor}
\usepackage{txfonts}
\usepackage{listings}
\usepackage{enumitem}
\usepackage{mathtools}
\usepackage{gensymb}
\usepackage{comment}
\usepackage[breaklinks=true]{hyperref}
\usepackage{tkz-euclide} 
\usepackage{listings}
\usepackage{gvv}                                        
%\def\inputGnumericTable{}                                 
\usepackage[latin1]{inputenc}                                
\usepackage{color}                                            
\usepackage{array}                                            
\usepackage{longtable}                                       
\usepackage{calc}                                             
\usepackage{multirow}                                         
\usepackage{hhline}                                           
\usepackage{ifthen}                                           
\usepackage{lscape}
\usepackage{tabularx}
\usepackage{array}
\usepackage{float}
\newtheorem{theorem}{Theorem}[section]
\newtheorem{problem}{Problem}
\newtheorem{proposition}{Proposition}[section]
\newtheorem{lemma}{Lemma}[section]
\newtheorem{corollary}[theorem]{Corollary}
\newtheorem{example}{Example}[section]
\newtheorem{definition}[problem]{Definition}
\newcommand{\BEQA}{\begin{eqnarray}}
\newcommand{\EEQA}{\end{eqnarray}}
\newcommand{\define}{\stackrel{\triangle}{=}}
\theoremstyle{remark}
\newtheorem{rem}{Remark}

% Marks the beginning of the document
\begin{document}
\bibliographystyle{IEEEtran}
\vspace{3cm}

\title{\textbf{22-07-2021 Shift-2(16-30)}}
\author{AI24BTECH11012- Pushkar Gudla}
\maketitle
\bigskip

\renewcommand{\thefigure}{\theenumi}
\renewcommand{\thetable}{\theenumi}
\setlength{\columnsep}{2.5em}

\begin{enumerate}
    \item The number of solutions of $ \sin^7 x + \cos^7 x = 1 $, where $ x \in [0, 4\pi] $, is equal to:
    \begin{enumerate}
        \item $11$
        \item $7$
        \item $5$
        \item $9$
    \end{enumerate}

    \item If the domain of the function $f(x) = \frac{\cos^{-1}\sqrt{x^2 - x + 1}}{\sqrt{\sin^{-1}\brak{\frac{2x-1}{2}}}}$ is the interval $ (\alpha, \beta] $, then $ \alpha + \beta $ is equal to:
    \begin{enumerate}
        \item $ \frac{3}{2} $
        \item $2$
        \item $ \frac{1}{2} $
        \item $1$
    \end{enumerate}

    \item Let $ f : \mathbb{R} \to \mathbb{R} $ be defined as:
    $f(x) =
    \begin{cases}
        \frac{x^3}{(1-\cos 2x)^2}\log_e\brak{\frac{1+2xe^{-2x}}{(1-xe^{-x})^2}} & , x \neq 0 \\
        \alpha & , x = 0
    \end{cases}$\\
    If $ f $ is continuous at $ x = 0 $, then $ \alpha $ is equal to:
    \begin{enumerate}
        \item $1$
        \item $3$
        \item $0$
        \item $2$
    \end{enumerate}

    \item Let a line $ L : 2x + y = k $, $ k > 0 $, be a tangent to the hyperbola $ x^2 - y^2 = 3 $. If $ L $ is also a tangent to the parabola $ y^2 = \alpha x $, then $ \alpha $ is equal to:
    \begin{enumerate}
        \item $12$
        \item $-12$
        \item $24$
        \item $-24$
    \end{enumerate}

    \item Let $E_1 : \frac{x^2}{a^2} + \frac{y^2}{b^2} = 1, \quad a > b$. Let $ E_2 $ be another ellipse such that it touches the endpoints of the major axis of $ E_1 $, and the foci of $ E_2 $ are the endpoints of the minor axis of $ E_1 $. If $ E_1 $ and $ E_2 $ have the same eccentricities, then its value is:
    \begin{enumerate}
        \item $ \frac{-1 + \sqrt{5}}{2} $
        \item $ \frac{-1 + \sqrt{8}}{2} $
        \item $ \frac{-1 + \sqrt{3}}{2} $
        \item $ \frac{-1 + \sqrt{6}}{2} $
    \end{enumerate}

    \item Let $ A = \{0, 1, 2, 3, 4, 5, 6, 7\} $. The number of bijective functions $ f : A \to A $ such that $ f(1) + f(2) = 3 - f(3) $ is equal to

    \item If the digits are not allowed to repeat in any number formed by using the digits 0, 2, 4, 6, 8, then the number of all numbers greater than 10,000 is equal to \rule{2.5cm}{0.4pt}.

    \item Let $A = 
    \begin{pmatrix}
    0 & 1 & 0 \\
    1 & 0 & 0 \\
    0 & 0 & 1
    \end{pmatrix}.
    $
    The number of $ 3 \times 3 $ matrices $ B $ with entries from the set $ \{1, 2, 3, 4, 5\} $ and satisfying $ AB = BA $ is equal to \rule{2.5cm}{0.4pt}.

    \item Consider the following frequency distribution:\\
    $ \begin{array}{c c c c c c c}
\text{\textbf{Class:}} & 0-6 & 6-12 & 12-18 & 18-24 & 24-30 \\
\text{\textbf{Frequency:}} & a & b & 12 & 9 & 5
\end{array}$\\
    If the mean is $ \frac{309}{22} $ and the median is 14, then the value $ (a - b)^2 $ is equal to \rule{2.5cm}{0.4pt}.

    \item The sum of all the elements in the set $ \{n \in \{1, 2, \ldots, 100\} \mid \text{H.C.F. of } n \text{ and } 2040 = 1 \} $ is equal to \rule{2.5cm}{0.4pt}.

    \item The area (in square units) of the region bounded by the curves $ x^2 + 2y - 1 = 0 $, $ y^2 + 4x - 4 = 0 $, and $ y^2 - 4x - 4 = 0 $ in the upper half-plane is equal to \rule{2.5cm}{0.4pt}.

    \item Let $ f : \mathbb{R} \to \mathbb{R} $ be a function defined as:\\
    $
    f(x) =
    \begin{cases}
    3\brak{1-\frac{|x|}{2}} & \text{if } |x| \leq 2 \\
    0 & \text{if } |x| > 2
    \end{cases}.
    $\\
    Let $ g : \mathbb{R} \to \mathbb{R} $ be given by $ g(x) = f(x+2) - f(x-2) $. If $ n $ and $ m $ denote the number of points in $ \mathbb{R} $ where $ g $ is not continuous and not differentiable, respectively, then $ n + m $ is equal to \rule{2.5cm}{0.4pt}.

    \item If the constant term in the binomial expansion of $ \brak{2x^{r} + \frac{1}{x^2}}^{10} $ is 180, then $ r $ is equal to \rule{2.5cm}{0.4pt}.

    \item Let $ y = y(x) $ be the solution of the differential equation
    $
    \brak{\brak{x+2}e^{\frac{y+1}{x+2}}+\brak{y+1}} dx = \brak{x+2} dy,
    $
    $ y(1) = 1 $. If the domain of $ y = y(x) $ is an open interval $ (\alpha, \beta) $, then $ |\alpha + \beta| $ is equal to \rule{2.5cm}{0.4pt}.
    \item The number of elements in the set $ \{n \in \{1, 2, 3, \ldots, 100\} \mid 11^n > 10^n + 9^n \} $ is \rule{2.5cm}{0.4pt}.
\end{enumerate}

\end{document}



