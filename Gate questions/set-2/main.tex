%iffalse
\let\negmedspace\undefined
\let\negthickspace\undefined
\documentclass[journal,12pt,onecolumn]{IEEEtran}
\usepackage{cite}
\usepackage{amsmath,amssymb,amsfonts,amsthm}
\usepackage{algorithmic}
\usepackage{multicol}
\usepackage{circuitikz}
\usepackage{tikz}
\usepackage{graphicx}
\usepackage{textcomp}
\usepackage{xcolor}
\usepackage{txfonts}
\usepackage{listings}
\usepackage{enumitem}
\usepackage{mathtools}
\usepackage{gensymb}
\usepackage{comment}
\usepackage[breaklinks=true]{hyperref}
\usepackage{tkz-euclide} 
\usepackage{listings}
\usepackage{gvv}                                        
%\def\inputGnumericTable{}                                 
\usepackage[latin1]{inputenc}                                
\usepackage{color}                                            
\usepackage{array}                                            
\usepackage{longtable}                                       
\usepackage{calc}                                             
\usepackage{multirow}                                         
\usepackage{hhline}                                           
\usepackage{ifthen}                                           
\usepackage{lscape}
\usepackage{tabularx}
\usepackage{array}
\usepackage{float}
\newtheorem{theorem}{Theorem}[section]
\newtheorem{problem}{Problem}
\newtheorem{proposition}{Proposition}[section]
\newtheorem{lemma}{Lemma}[section]
\newtheorem{corollary}[theorem]{Corollary}
\newtheorem{example}{Example}[section]
\newtheorem{definition}[problem]{Definition}
\newcommand{\BEQA}{\begin{eqnarray}}
\newcommand{\EEQA}{\end{eqnarray}}
\newcommand{\define}{\stackrel{\triangle}{=}}
\theoremstyle{remark}
\newtheorem{rem}{Remark}

% Marks the beginning of the document
\begin{document}
\bibliographystyle{IEEEtran}
\vspace{3cm}

\title{\textbf{AE-2015 14-26}}
\author{AI24BTECH11012- Pushkar Gudla}
\maketitle
\bigskip

\renewcommand{\thefigure}{\theenumi}
\renewcommand{\thetable}{\theenumi}
\setlength{\columnsep}{2.5em}

\begin{enumerate}
    \item A cantilever with thin-walled channel cross section is subjected to a lateral force at its shear center. The cantilever undergoes.
    \begin{enumerate}
    \item bending without twisting
    \item bending and twisting
    \item neither bending nor twisting
    \item twisting without bending
    \end{enumerate}
  \item The two non-zero principal stresses at a point in a thin plate are $\sigma_1=25$ MPa and $\sigma_2=-25$ MPa. The maximum shear stress(in MPa) at this point is \rule{2.5cm}{0.4pt}.
  \item Consider the density and altitude at the base of an isothermal layer in the standard atmosphere to be $\rho_1$ and $h_1$, respectively. The density variation($\rho$ versus $h$) in that layer is governed by ($R:$specific gas constant, $T:$temperature, $g_o$acceleration due to gravity at sea level)
    \begin{enumerate}
    \item $\frac{\rho}{\rho_1}=e^{\frac{g_o}{RT}\brak{h-h_1}}$
    \item $\frac{\rho}{\rho_1}=e^{\frac{g_o}{RT}\brak{h_1-h}}$
    \item $\frac{\rho}{\rho_1}=e^{\frac{RT}{g_o}\brak{h-h_1}}$
    \item $\frac{\rho}{\rho_1}=e^{\frac{RT}{g_o}\brak{h_1-h}}$
    \end{enumerate}
  \item For constant free stream velocity and density, a change in lift for a large aspect ratio straight wing with thin cambered section at small angles of attack, leads to:
  \begin{enumerate}
  \item a shift of the aerodynamic center and no shift of the center of pressure
  \item a shift of the center of pressure and no shift of the aerodynamic center
  \item shift of both the aerodynamic center and center of pressure
  \item no shift either of the aerodynamic center or of the center of pressure
  \end{enumerate}
 \item Which of the following modes of a stable aircraft has non-oscillatory response characteristics?
   \begin{enumerate}
  \item Short period
  \item Phugoid
  \item Dutch roll
  \item Spiral
  \end{enumerate}
 \item As a candidate for vertical tail which one of the following airfoil sections is appropriate?
   \begin{enumerate}
  \item NACA 0012
  \item NACA 2312
  \item NACA 23012
  \item Clarke Y profile
  \end{enumerate}
 \item The primary purpose of a trailing edge flap is to
   \begin{enumerate}
  \item avoid flow seperation
  \item increase $C_{l,max}$
  \item reduce wave drag
  \item reduce induced drag
  \end{enumerate}
 \item Which one of the following aero engines has the highest propulsive efficiency?
   \begin{enumerate}
  \item Turbojet engine without afterburner
  \item Turbojet engine with afterburner 
  \item Turbofan engine
  \item Ramjet engine
  \end{enumerate}
 \item The stoichiometric fuel-to-air ratio in an aircraft engine combustor varies with the compressor pressure ratio as follows:
   \begin{enumerate}
  \item increases linearly
  \item decreases linearly 
  \item is independent
  \item increases nonlinearly
  \end{enumerate}
 \item A rocket engine produces a total impulse of $112 kN.s$ in a burn time period of 3.5 minutes with a propellant mass flow rate of $0.25 kg/s$. The effective exhaust velocity (in $m/s$) of gas ejecting from the engine is \rule{2.5cm}{0.4pt}.
\item The function $y=x^3-x$ has
\begin{enumerate}
\item no inflection point
\item one inflection point
\item two inflection points
\item three inflection points
\end{enumerate}
\item A $0.5 kg$ mass is suspended vertically from a point fixed on the Earth by a spring having a stiffness of $5 N/mm$. The static displacement(in $mm$) of the mass is \rule{2.5cm}{0.4pt}.

\item A slender structure is subjected to four different loading (I, II, III and IV) as shown below(Figures not to scale). Which pair pf cases results in identical stress distribution at sections S-S located far away from both ends?

[I]


\begin{figure}[!ht]
\centering
\resizebox{0.4\textwidth}{!}{%
\begin{circuitikz}
\tikzstyle{every node}=[font=\small]
\draw (1.75,16.25) to[short] (2,16.25);
\draw (1.75,15.75) to[short] (2,15.75);
\draw (2,16.25) node[ieeestd nand port, anchor=in 1, scale=0.89](port){} (port.out) to[short] (3.75,16);
\draw (1.75,14) to[short] (2,14);
\draw (1.75,13.5) to[short] (2,13.5);
\draw (2,14) node[ieeestd nand port, anchor=in 1, scale=0.89](port){} (port.out) to[short] (3.75,13.75);
\draw (3.75,16) to[short] (4.25,16);
\draw (3.75,13.75) to[short] (4.25,13.75);
\draw (4.25,16) to[short] (4.25,15.25);
\draw (4.25,15.25) to[short] (4.75,15.25);
\draw (4.25,14.75) to[short] (4.25,13.75);
\draw (4.25,14.75) to[short] (4.75,14.75);
\draw (4.5,15.25) to[short] (4.75,15.25);
\draw (4.5,14.75) to[short] (4.75,14.75);
\draw (4.75,15.25) node[ieeestd nand port, anchor=in 1, scale=0.89](port){} (port.out) to[short] (6.5,15);
\draw (6.5,15) to[short] (7,15);
\node [font=\small] at (1.5,16.25) {A};
\node [font=\small] at (1.5,15.75) {B};
\node [font=\small] at (1.5,14) {C};
\node [font=\small] at (1.5,13.5) {D};
\node [font=\small] at (7.25,15) {Y};
\end{circuitikz}
}%

\label{fig:my_label}
\end{figure}
 % Adjusts the width to half of the text width


[II]


\begin{figure}[!ht]
\centering
\resizebox{0.4\textwidth}{!}{%
\begin{circuitikz}
\tikzstyle{every node}=[font=\small]
\draw (4.5,14.25) node[op amp,scale=1, yscale=-1 ] (opamp2) {};
\draw (opamp2.+) to[short] (3,14.75);
\draw  (opamp2.-) to[short] (3,13.75);
\draw (5.7,14.25) to[short](6,14.25);
\draw (0.75,14.75) to[american voltage source] (0.75,12.25);
\draw (0.75,12.25) to (0.75,12) node[ground]{};
\draw (2.75,16.75) to[short] (2.75,14.75);
\draw (2.75,16.75) to[european resistor] (6,16.75);
\draw (3,12.25) to[R] (6,12.25);
\draw (6,16.75) to[short] (6,12.25);
\draw (3,13.75) to[short] (3,12);
\draw (3,12.25) to[R] (3,10.5);
\draw (3,10.5) to (3,10) node[ground]{};
\node [font=\small] at (0,13.5) {$V_{in}$};
\node [font=\small] at (1.5,15.25) {$i_{in}$};
\node [font=\small] at (2.25,11.25) {$R_2$};
\node [font=\small] at (4.5,11.75) {$R_1$};
\node [font=\small] at (6.5,14.25) {$V_o$};
\node [font=\small] at (4.5,16.25) {Z};
\draw [->, >=Stealth] (0.75,14.75) -- (1.75,14.75);
\draw (1.75,14.75) to[short] (3,14.75);
\end{circuitikz}
}%

\label{fig:my_label}
\end{figure}
 % Adjusts the width to half of the text width


[III]


\begin{circuitikz}
\tikzstyle{every node}=[font=\small]
\draw  (1.75,15.75) rectangle (6.5,14);
\draw [short] (1.75,15.75) -- (1.25,15.25);
\draw [short] (1.75,15.25) -- (1.25,14.75);
\draw [short] (1.75,14.5) -- (1.25,14);
\draw [dashed] (3.25,16.25) -- (3.25,13.5);
\node [font=\small] at (3,13.75) {S};
\node [font=\small] at (3,16) {S};
\draw [->, >=Stealth] (6.5,15.75) -- (7.75,15.75);
\draw [->, >=Stealth] (6.5,15.25) -- (7.25,15.25);
\node [font=\small] at (8,15.25) {$\frac{P}{8}$};
\node [font=\small] at (8,14.75) {$\frac{P}{8}$};
\node [font=\small] at (8.20,16) {$\frac{3P}{8}$};
\node [font=\small] at (8.20,14) {$\frac{3P}{8}$};
\draw [->, >=Stealth] (7,14.75) -- (6.5,14.75);
\draw [->, >=Stealth] (7.75,14.25) -- (6.5,14.25);
\end{circuitikz}
 


[IV]


\begin{figure}[!ht]
\centering
\resizebox{0.4\textwidth}{!}{%
\begin{circuitikz}
\tikzstyle{every node}=[font=\small]
\begin{scope}[rotate around={2:(-0.25,13.25)}]
\draw[domain=-0.25:7,samples=100,smooth] plot (\x,{1*sin(1.7*\x r +0.25 r ) +13.25});
\end{scope}
\draw [->, >=Stealth, dashed] (-0.25,13.25) -- (7.75,13.25);
\draw [->, >=Stealth, dashed] (-0.25,12.5) -- (-0.25,15.25);
\draw [short] (-0.25,13.25) -- (0.25,13.25);
\draw [short] (0.25,13.25) -- (0.25,13.5);
\draw [short] (0.25,13.5) -- (1.75,13.5);
\draw [short] (1.75,13.5) -- (1.75,13.25);
\draw [short] (1.75,13.25) -- (2.25,13.25);
\draw [short] (2.25,13.25) -- (2.25,13);
\draw [short] (2.25,13) -- (3.25,13);
\draw [short] (3.25,13.25) -- (3.5,13.25);
\draw [short] (3.5,13.25) -- (3.5,13);
\draw [short] (3.25,13) -- (3.5,13);
\draw [short] (3.5,13.25) -- (4,13.25);
\draw [short] (4,13.25) -- (4,13.5);
\draw [short] (4,13.5) -- (5.25,13.5);
\draw [short] (5.25,13.5) -- (5.25,13.25);
\draw [short] (5.25,13.25) -- (5.75,13.25);
\draw [short] (5.75,13.25) -- (5.75,13);
\draw [short] (5.75,13) -- (7,13);
\draw [short] (7,13) -- (7,13.25);
\node [font=\small] at (-0.5,13.25) {0};
\node [font=\small] at (0.5,13) {$30\degree$};
\node [font=\small] at (1.75,14.25) {$V_m\sin wt$};
\node [font=\small] at (2.25,11) {$210\degree$};
\node [font=\small] at (0.5,15.5) {voltage and current};
\node [font=\small] at (0.75,13.75) {$I_o$};
\node [font=\small] at (1.75,12) {$180\degree$};
\node [font=\small] at (7.5,12.75) {$wt$};
\draw [->, >=Stealth] (1,14) -- (1,13.5);
\draw [->, >=Stealth] (1.25,12.75) -- (1.25,13.25);
\draw [->, >=Stealth] (2.75,13.75) -- (2.75,13.25);
\draw [->, >=Stealth] (2.75,12.75) -- (2.75,13);
\node [font=\small] at (3,13.5) {$I_o$};
\draw [dashed] (1.75,13.25) -- (1.75,12.25);
\draw [dashed] (2.25,13) -- (2.25,11);
\end{circuitikz}
}%

\label{fig:my_label}
\end{figure}
 

\begin{enumerate}
\item I and II
\item II and III
\item III and IV
\item IV and I
\end{enumerate}
\end{enumerate} 

\end{document}

