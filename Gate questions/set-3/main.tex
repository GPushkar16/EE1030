%iffalse
\let\negmedspace\undefined
\let\negthickspace\undefined
\documentclass[journal,12pt,onecolumn]{IEEEtran}
\usepackage{cite}
\usepackage{amsmath,amssymb,amsfonts,amsthm}
\usepackage{algorithmic}
\usepackage{multicol}
\usepackage{circuitikz}
\usepackage{tikz}
\usepackage{graphicx}
\usepackage{textcomp}
\usepackage{xcolor}
\usepackage{txfonts}
\usepackage{listings}
\usepackage{enumitem}
\usepackage{mathtools}
\usepackage{gensymb}
\usepackage{comment}
\usepackage[breaklinks=true]{hyperref}
\usepackage{tkz-euclide} 
\usepackage{listings}
\usepackage{gvv}                                        
%\def\inputGnumericTable{}                                 
\usepackage[latin1]{inputenc}                                
\usepackage{color}                                            
\usepackage{array}                                            
\usepackage{longtable}                                       
\usepackage{calc}                                             
\usepackage{multirow}                                         
\usepackage{hhline}                                           
\usepackage{ifthen}                                           
\usepackage{lscape}
\usepackage{tabularx}
\usepackage{array}
\usepackage{float}
\newtheorem{theorem}{Theorem}[section]
\newtheorem{problem}{Problem}
\newtheorem{proposition}{Proposition}[section]
\newtheorem{lemma}{Lemma}[section]
\newtheorem{corollary}[theorem]{Corollary}
\newtheorem{example}{Example}[section]
\newtheorem{definition}[problem]{Definition}
\newcommand{\BEQA}{\begin{eqnarray}}
\newcommand{\EEQA}{\end{eqnarray}}
\newcommand{\define}{\stackrel{\triangle}{=}}
\theoremstyle{remark}
\newtheorem{rem}{Remark}

% Marks the beginning of the document
\begin{document}
\bibliographystyle{IEEEtran}
\vspace{3cm}

\title{\textbf{AE-2015 14-26}}
\author{AI24BTECH11012- Pushkar Gudla}
\maketitle
\bigskip

\renewcommand{\thefigure}{\theenumi}
\renewcommand{\thetable}{\theenumi}
\setlength{\columnsep}{2.5em}

\begin{enumerate}
    \item In the logic circuit shown in the figure, Y is given by
    \begin{figure}[!ht]
\centering
\resizebox{0.4\textwidth}{!}{%
\begin{circuitikz}
\tikzstyle{every node}=[font=\small]
\draw (1.75,16.25) to[short] (2,16.25);
\draw (1.75,15.75) to[short] (2,15.75);
\draw (2,16.25) node[ieeestd nand port, anchor=in 1, scale=0.89](port){} (port.out) to[short] (3.75,16);
\draw (1.75,14) to[short] (2,14);
\draw (1.75,13.5) to[short] (2,13.5);
\draw (2,14) node[ieeestd nand port, anchor=in 1, scale=0.89](port){} (port.out) to[short] (3.75,13.75);
\draw (3.75,16) to[short] (4.25,16);
\draw (3.75,13.75) to[short] (4.25,13.75);
\draw (4.25,16) to[short] (4.25,15.25);
\draw (4.25,15.25) to[short] (4.75,15.25);
\draw (4.25,14.75) to[short] (4.25,13.75);
\draw (4.25,14.75) to[short] (4.75,14.75);
\draw (4.5,15.25) to[short] (4.75,15.25);
\draw (4.5,14.75) to[short] (4.75,14.75);
\draw (4.75,15.25) node[ieeestd nand port, anchor=in 1, scale=0.89](port){} (port.out) to[short] (6.5,15);
\draw (6.5,15) to[short] (7,15);
\node [font=\small] at (1.5,16.25) {A};
\node [font=\small] at (1.5,15.75) {B};
\node [font=\small] at (1.5,14) {C};
\node [font=\small] at (1.5,13.5) {D};
\node [font=\small] at (7.25,15) {Y};
\end{circuitikz}
}%

\label{fig:my_label}
\end{figure}

    \begin{enumerate}
    \item Y=ABCD
    \item Y=(A+B)(C+D)
    \item Y=A+B+C+D
    \item Y=AB+CD
    \end{enumerate}
    
    \item The op-amp shown in the figure is ideal. The input impedance $\frac{v_{in}}{i_{in}}$ is given by
    \begin{figure}[!ht]
\centering
\resizebox{0.4\textwidth}{!}{%
\begin{circuitikz}
\tikzstyle{every node}=[font=\small]
\draw (4.5,14.25) node[op amp,scale=1, yscale=-1 ] (opamp2) {};
\draw (opamp2.+) to[short] (3,14.75);
\draw  (opamp2.-) to[short] (3,13.75);
\draw (5.7,14.25) to[short](6,14.25);
\draw (0.75,14.75) to[american voltage source] (0.75,12.25);
\draw (0.75,12.25) to (0.75,12) node[ground]{};
\draw (2.75,16.75) to[short] (2.75,14.75);
\draw (2.75,16.75) to[european resistor] (6,16.75);
\draw (3,12.25) to[R] (6,12.25);
\draw (6,16.75) to[short] (6,12.25);
\draw (3,13.75) to[short] (3,12);
\draw (3,12.25) to[R] (3,10.5);
\draw (3,10.5) to (3,10) node[ground]{};
\node [font=\small] at (0,13.5) {$V_{in}$};
\node [font=\small] at (1.5,15.25) {$i_{in}$};
\node [font=\small] at (2.25,11.25) {$R_2$};
\node [font=\small] at (4.5,11.75) {$R_1$};
\node [font=\small] at (6.5,14.25) {$V_o$};
\node [font=\small] at (4.5,16.25) {Z};
\draw [->, >=Stealth] (0.75,14.75) -- (1.75,14.75);
\draw (1.75,14.75) to[short] (3,14.75);
\end{circuitikz}
}%

\label{fig:my_label}
\end{figure}

    \begin{enumerate}
    \item Z$\frac{R_1}{R_2}$
    \item -Z$\frac{R_2}{R_1}$
    \item Z
    \item -Z$\frac{R_1}{R_1+R_2}$
    \end{enumerate}
    
    \item A continuous-time input signal $x{t}$ is an eigenfunction of an LTI system, if the output is
    \begin{enumerate}
    \item $kx{t}$, where $k$ is an eigenvalue
    \item $ke^{iwt}x(t)$ where $k$ is an eigenvalue and $e^{iwt}$ is a complex exponential signal
    \item $x(t)e^{iwt}$, where $e^{iwt}$ is a complex signal
    \item $kH(w)$, where $k$ is an eigenvalue and $H(w)$ is a frequency response of the system
    \end{enumerate}
    
    \item Consider a non-singular $2x2$ square matrix $\vec{A}$. If $trace(\vec{A})=4$ and $trace{\vec{A}^2}=5$, the determinant of the matrix $\vec{A}$ is \rule{2.5cm}{0.4pt}\brak{up to 1 decimal place}.
    
    \item Let $f$ be a real-valued function of a real variable defined as $f(x)=x-[x]$, where $[x]$ denoted the largest integer less than or equal to $x$. The value of $\int_{0.25}^{1.25} f(x)dx$ is \rule{2cm}{0.4pt}\brak{up to 2 decimal places}.
    
    \item In the two-port network shown, the $h_{11}$ parameter \brak{where, $h_{11}=\frac{V_1}{I_1}$, when $V_2=0$} in ohms is \rule{2.5cm}{0.4pt}\brak{up to 2 decimal places}.
    \begin{circuitikz}
\tikzstyle{every node}=[font=\small]
\draw  (1.75,15.75) rectangle (6.5,14);
\draw [short] (1.75,15.75) -- (1.25,15.25);
\draw [short] (1.75,15.25) -- (1.25,14.75);
\draw [short] (1.75,14.5) -- (1.25,14);
\draw [dashed] (3.25,16.25) -- (3.25,13.5);
\node [font=\small] at (3,13.75) {S};
\node [font=\small] at (3,16) {S};
\draw [->, >=Stealth] (6.5,15.75) -- (7.75,15.75);
\draw [->, >=Stealth] (6.5,15.25) -- (7.25,15.25);
\node [font=\small] at (8,15.25) {$\frac{P}{8}$};
\node [font=\small] at (8,14.75) {$\frac{P}{8}$};
\node [font=\small] at (8.20,16) {$\frac{3P}{8}$};
\node [font=\small] at (8.20,14) {$\frac{3P}{8}$};
\draw [->, >=Stealth] (7,14.75) -- (6.5,14.75);
\draw [->, >=Stealth] (7.75,14.25) -- (6.5,14.25);
\end{circuitikz}

    
    \item The series impedance matrix of a short three-phase transmission line in phase coordinates
$
\begin{bmatrix}
Z_s & Z_m & Z_m \\
Z_m & Z_s & Z_m \\
Z_m & Z_m & Z_s 
\end{bmatrix}
$
is given. If the positive sequence impedance is $ (1 + j10) \Omega $, and the zero sequence impedance is $ (4 + j31) \Omega$, then the imaginary part of $ Z_m $ (in $ \Omega $) is\rule{2.5cm}{0.4pt} \brak{up to 2 decimal places}.

\item The positive, negative and zero sequence impedances of a 125 MVA, three-phase, 15.5 kV, star-grounded, 50 Hz generator are $ j0.1 $ pu, $ j0.05 $ pu, and $ j0.01 $ pu respectively on the machine rating base. The machine is unloaded and working at the rated terminal voltage. If the grounding impedance of the generator is $ j0.01 $ pu, then the magnitude of fault current for a $ b $-phase to ground fault (in kA) is \rule{2cm}{0.4pt} (up to 2 decimal places).

\item A $ 1000 \times 1000$ bus admittance matrix for an electric power system has 8000 non-zero elements. The minimum number of branches (transmission lines and transformers) in this system are \rule{2cm}{0.4pt}(up to 2 decimal places).

\item The waveform of the current drawn by a semi-converter from a sinusoidal AC voltage source is shown in the figure. If $I_o=20$A, the rms value of fundamental component of the current is \rule{2cm}{0.4pt}A\brak{up to 2 decimal places}.
\begin{figure}[!ht]
\centering
\resizebox{0.4\textwidth}{!}{%
\begin{circuitikz}
\tikzstyle{every node}=[font=\small]
\begin{scope}[rotate around={2:(-0.25,13.25)}]
\draw[domain=-0.25:7,samples=100,smooth] plot (\x,{1*sin(1.7*\x r +0.25 r ) +13.25});
\end{scope}
\draw [->, >=Stealth, dashed] (-0.25,13.25) -- (7.75,13.25);
\draw [->, >=Stealth, dashed] (-0.25,12.5) -- (-0.25,15.25);
\draw [short] (-0.25,13.25) -- (0.25,13.25);
\draw [short] (0.25,13.25) -- (0.25,13.5);
\draw [short] (0.25,13.5) -- (1.75,13.5);
\draw [short] (1.75,13.5) -- (1.75,13.25);
\draw [short] (1.75,13.25) -- (2.25,13.25);
\draw [short] (2.25,13.25) -- (2.25,13);
\draw [short] (2.25,13) -- (3.25,13);
\draw [short] (3.25,13.25) -- (3.5,13.25);
\draw [short] (3.5,13.25) -- (3.5,13);
\draw [short] (3.25,13) -- (3.5,13);
\draw [short] (3.5,13.25) -- (4,13.25);
\draw [short] (4,13.25) -- (4,13.5);
\draw [short] (4,13.5) -- (5.25,13.5);
\draw [short] (5.25,13.5) -- (5.25,13.25);
\draw [short] (5.25,13.25) -- (5.75,13.25);
\draw [short] (5.75,13.25) -- (5.75,13);
\draw [short] (5.75,13) -- (7,13);
\draw [short] (7,13) -- (7,13.25);
\node [font=\small] at (-0.5,13.25) {0};
\node [font=\small] at (0.5,13) {$30\degree$};
\node [font=\small] at (1.75,14.25) {$V_m\sin wt$};
\node [font=\small] at (2.25,11) {$210\degree$};
\node [font=\small] at (0.5,15.5) {voltage and current};
\node [font=\small] at (0.75,13.75) {$I_o$};
\node [font=\small] at (1.75,12) {$180\degree$};
\node [font=\small] at (7.5,12.75) {$wt$};
\draw [->, >=Stealth] (1,14) -- (1,13.5);
\draw [->, >=Stealth] (1.25,12.75) -- (1.25,13.25);
\draw [->, >=Stealth] (2.75,13.75) -- (2.75,13.25);
\draw [->, >=Stealth] (2.75,12.75) -- (2.75,13);
\node [font=\small] at (3,13.5) {$I_o$};
\draw [dashed] (1.75,13.25) -- (1.75,12.25);
\draw [dashed] (2.25,13) -- (2.25,11);
\end{circuitikz}
}%

\label{fig:my_label}
\end{figure}


\item A separately excited dc motor has an armature resistance $R_a=0.05\ohm$. The field excitation is kept constant. At an armature voltage of 100V, the motor produces a torque of 500 Nm at zero speed. Neglecting all mechanical losses, the no-load speed of the motor(in radian/s) for an armature voltage of 150 V is \rule{2cm}{0.4pt}\brak{up to 2 decimal places}.

\item Consider a unity feedback system with forward transfer function given by 
\[
G(s)=\frac{1}{\brak{s+1}\brak{s+2}}
\]
The steady-state error in the output of the system for a unit-step input is \rule{2cm}{0.4pt}\brak{up to 2 decimal places}.

\item A transformer with toroidal core of permeabiltiy $\mu$ is shown in the figure. Assuming uniform flux density across the circular core cross-section of radius $r<R$, and neglecting any leakage flux, the best estimate for the mean radius $R$ is

\begin{figure}[!ht]
\centering
\resizebox{0.4\textwidth}{!}{%
\begin{circuitikz}
\tikzstyle{every node}=[font=\small]
\draw  (2.75,16) circle (2cm);
\draw  (2.75,16) circle (1cm);
\draw [ dashed] (2.75,16) circle (1.5cm);
\draw  (2.75,17.5) ellipse (0.25cm and 0.5cm);
\draw [short] (4.5,17) -- (7.25,17);
\draw [short] (4.5,15) -- (7.25,15);
\draw [short] (1,17) -- (-0.75,17);
\draw [short] (1,15) -- (-0.75,15);
\draw (-0.75,17) to[sinusoidal voltage source, sources/symbol/rotate=auto] (-0.75,15);
\node [font=\small] at (-0.75,17.25) {+};
\node [font=\small] at (-0.75,14.75) {-};
\node [font=\small] at (7,17.25) {+};
\node [font=\small] at (7,14.75) {-};
\node at (2.75,17.5) [circ] {};
\draw [->, >=Stealth] (2.75,17.5) -- (2.75,18);
\node [font=\small] at (3,18.25) {r};
\node at (2.75,16) [circ] {};
\draw [->, >=Stealth] (2.75,16) -- (2.75,15);
\draw [short] (0.75,16) -- (1.75,16);
\draw [short] (3.75,16) -- (4.75,16);
\draw [short] (1,15) .. controls (1.75,15) and (1.75,15) .. (2,15.25);
\draw [short] (4.5,17) .. controls (4,17) and (4,17) .. (3.5,16.75);
\draw [short] (4.75,16.5) .. controls (4.25,16.5) and (4.25,16.5) .. (3.75,16.25);
\draw [short] (4.5,15) .. controls (4,14.75) and (4,15) .. (3.5,15.25);
\draw [short] (4.75,15.5) .. controls (4.25,15.5) and (4.25,15.5) .. (3.75,15.75);
\node [font=\small] at (-1.5,15.5) {$v_p=V\cos wt$};
\node [font=\small] at (0.25,17.5) {$i_p=I\sin wt$};
\node [font=\small] at (5,16) {$N_s$};
\node [font=\small] at (0.25,16) {$N_p$};
\node [font=\small] at (5.5,17.5) {$i_s=0$};
\node [font=\small] at (3,15.5) {R};
\node [font=\small] at (7,16) {$V_s$};
\draw [->, >=Stealth] (-0.25,17.25) -- (0.75,17.25);
\draw [->, >=Stealth] (5.25,17.25) -- (6.25,17.25);
\draw [short] (0.75,15.5) .. controls (1.5,15.5) and (1.25,15.5) .. (1.75,15.75);
\draw [short] (1,17) .. controls (1.75,17) and (1.5,17) .. (2,16.75);
\draw [short] (0.75,16.5) .. controls (1.25,16.5) and (1.25,16.5) .. (1.75,16.25);
\end{circuitikz}
}%

\label{fig:my_label}
\end{figure}

\begin{enumerate}
\item $\frac{\mu Vr^2N_p^2w}{I}$
\item $\frac{\mu Ir^2N_pN_sw}{V}$
\item $\frac{\mu Vr^2N_p^2w}{2I}$
\item $\frac{\mu Ir^2N_p^2w}{2V}$
\end{enumerate}
\end{enumerate}
\end{document}

