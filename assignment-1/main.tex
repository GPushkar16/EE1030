%iffalse
\let\negmedspace\undefined
\let\negthickspace\undefined
\documentclass[journal,12pt,onecolumn]{IEEEtran}
\usepackage{cite}
\usepackage{amsmath,amssymb,amsfonts,amsthm}
\usepackage{algorithmic}
\usepackage{multicol}
\usepackage{graphicx}
\usepackage{textcomp}
\usepackage{xcolor}
\usepackage{txfonts}
\usepackage{listings}
\usepackage{enumitem}
\usepackage{mathtools}
\usepackage{gensymb}
\usepackage{comment}
\usepackage[breaklinks=true]{hyperref}
\usepackage{tkz-euclide} 
\usepackage{listings}
\usepackage{gvv}                                        
%\def\inputGnumericTable{}                                 
\usepackage[latin1]{inputenc}                                
\usepackage{color}                                            
\usepackage{array}                                            
\usepackage{longtable}                                       
\usepackage{calc}                                             
\usepackage{multirow}                                         
\usepackage{hhline}                                           
\usepackage{ifthen}                                           
\usepackage{lscape}
\usepackage{tabularx}
\usepackage{array}
\usepackage{float}
\newtheorem{theorem}{Theorem}[section]
\newtheorem{problem}{Problem}
\newtheorem{proposition}{Proposition}[section]
\newtheorem{lemma}{Lemma}[section]
\newtheorem{corollary}[theorem]{Corollary}
\newtheorem{example}{Example}[section]
\newtheorem{definition}[problem]{Definition}
\newcommand{\BEQA}{\begin{eqnarray}}
\newcommand{\EEQA}{\end{eqnarray}}
\newcommand{\define}{\stackrel{\triangle}{=}}
\theoremstyle{remark}
\newtheorem{rem}{Remark}

% Marks the beginning of the document
\begin{document}
\bibliographystyle{IEEEtran}
\vspace{3cm}

\title{\textbf{Assignment-1}}
\author{AI24BTECH11012- Pushkar Gudla}
\maketitle
\bigskip

\renewcommand{\thefigure}{\theenumi}
\renewcommand{\thetable}{\theenumi}
\setlength{\columnsep}{2.5em}
\section*{\textbf{SECTION-A}}
\section*{\textbf{F.}  Match the Following}
	\begin{enumerate}
		\item Match the following:

			\hfill{(2006-6M)}
			\begin{multicols}{2}
				\begin{enumerate}
				\item	$\int_{0}^{\frac{\pi}{2}} \brak{{\sin x}}^{\cos x} \brak{\cos x\cot x-\log\brak{\sin x}^{\sin x}} dx $
				\item Area bounded by $-4y^{2}=x$ and $x-1=-5y^{2}$
				\item Cosine of the angle of intersection of curves $y=3^{x-1}\log x$ and $y=x^{x}-1$ is
				\item Let $\frac{dy}{dx}=\frac{6}{x+y}$ where $y\brak{0}=0$ then value of $y$ when $x+y=6$ is
				\end{enumerate}
			\columnbreak
				\begin{enumerate}
					\item $1$
						
					\item $0$

					\item $6\ln(2)$

					\item $\frac{4}{3}$
						
				\end{enumerate}
				\end{multicols}
			\item Match the integrals in \textbf{Column I} with the values in \textbf{Column II} and indicate your answer by darkening the appropriate bubbles in the $4x4$ matrix given in the ORS. 

			\hfill{(2007-6M)}
			\begin{multicols}{2}
				\textbf{Column I}

				\begin{enumerate}
					\item $\int_{-1}^{1}\frac{dx}{1+x^{2}}$
					\item $\int_{0}^{1}\frac{dx}{\sqrt{1-x^{2}}}$
					\item $\int_{2}^{3}\frac{dx}{1-x^{2}}$
					\item $\int_{1}^{2}\frac{dx}{x\sqrt{x^{2}-1}}$
				\end{enumerate}
			\columnbreak
				\textbf{Column II}
				\begin{enumerate}
					\item $\frac{1}{2}\log\brak{\frac{2}{3}}$

					\item $2\log\brak{\frac{2}{3}}$

					\item $\frac{\pi}{3}$

					\item $\frac{\pi}{2}$
				\end{enumerate}
			\end{multicols}


		\item 
			\hfill{(JEE Adv. 2014)}
			\begin{multicols}{2}
				\textbf{List-I} 
				\begin{enumerate}
					\item The number of polynimials $f\brak{x}$ with non-negative integer coeffecients of $degree \leq 2$, satisfying $f\brak{0}=0$ and $\int_{0}^{1}f\brak{x}dx=1$, is
					\item The number of points in the interval $\left[-\sqrt{13},\sqrt{13}\right]$ at which $f\brak{x}= \sin x^{2}+\cos x^{2}$attains its maximum value is
					\item $\int_{-2}^{2}\frac{3x^{2}}{\brak{1+e^{x}}}dx$ equals
					\item $\frac{\brak{\int_{\frac{-1}{2}}^{\frac{1}{2}}\cos 2x\log\brak{\frac{1+x}{1-x}}dx}}{\brak{\int_{0}^{\frac{1}{2}}\cos 2x\log\brak{\frac{1+x}{1-x}}dx}}$
				\end{enumerate}
				\columnbreak
				\textbf{List-II}
				\begin{enumerate}
					\item $8$

					\item $2$

					\item $4$

					\item $0$
				\end{enumerate}
		\end{multicols}
			\textbf{   P Q R S}
			\begin{enumerate}
		
				\item $3 2 4 1$
				\item $3 2 1 4$
				\item $2 3 4 1$
				\item $2 3 1 4$
			\end{enumerate}
	\end{enumerate}
\section*{\textbf{Section-B} JEE MAIN/AIEEE }
\begin{enumerate}
	\item The area \brak{in sq. units} of the region $\{\brak{x,y}:y^{2}\geq 2x and x^{2}+y^{2}\leq 4x, x\geq0, y\geq0\}$ is:

		\hfill{[JEE M 2016]}
		\begin{enumerate}
			\item $\pi - \frac{4\sqrt{2}}{3}$

			\item $\frac{\pi}{2}-\frac{2\sqrt{2}}{3}$

			\item $\pi - \frac{4}{3}$

			\item $\pi - \frac{8}{3}$
		\end{enumerate}
	\item The area \brak{in sq. units} of the region $\{\brak{x,y}: x\geq0, x+y\leq3, x^{2}\leq4y and y\leq1+\sqrt{x}\}$ is:

		\hfill{[JEE M 2017]}
		\begin{enumerate}
			\item $\frac{5}{2}$

			\item $\frac{59}{12}$

			\item $\frac{3}{2}$

			\item $\frac{7}{3}$
		\end{enumerate}
	\item The integral $\int_{\frac{\pi}{4}}^{\frac{3\pi}{4}}\frac{dx}{1+\cos x}$ is equal to:

		\hfill{[JEE M 2017]}
		\begin{enumerate}
			\item $-1$
			\item $-2$
			\item $42$
			\item $4$
		\end{enumerate}
	\item Let $g\brak{x}=\cos x^{2}, f\brak{x}=\sqrt{x}, and \alpha, \beta (\alpha<\beta)$ be the roots of the quadratic equation $18x^{2}-9\pi x+\pi^{2}=0.$ Then the area \brak{in sq. units} bounded by the curve $y=\brak{gof}\brak{x}$ and the lines $x=\alpha, x=\beta and y=0,$ is:

		\hfill{[JEE M 2018]}
		\begin{enumerate}
			\item $\frac{1}{2}\brak{\sqrt{3}+1}$
			\item $\frac{1}{2}\brak{\sqrt{3}-\sqrt{2}}$
			\item $\frac{1}{2}\brak{\sqrt{2}-1}$
			\item $\frac{1}{2}\brak{\sqrt{3}-1}$
		\end{enumerate}
	\item The value of $\int_{\frac{-\pi}{2}}^{\frac{\pi}{2}}\frac{\sin^{2} x}{1+2^{x}}dx$ is:

		\hfill{[JEE M 2018]}
		\begin{enumerate}
			\item $\frac{\pi}{2}$
			\item $4\pi$
			\item $\frac{\pi}{4}$
			\item $\frac{\pi}{8}$
		\end{enumerate}
	\item The value of $\int_{0}^{\pi}\mid{\cos x}\mid^{3}dx$ is:

		\hfill{[JEE M 2019-9 Jan(M)]}
		\begin{enumerate}
			\item $0$
			\item $\frac{4}{3}$
			\item $\frac{2}{3}$
			\item $\frac{-2}{3}$
		\end{enumerate}
	\item The area \brak{in sq. units} bounded by the parabola $y=x^{2}-1,$ the tangent at the point \brak{2,3} to it and the y-axis is:

		\hfill{[JEE M 2019-9Jan(M)]}
		\begin{enumerate}
			\item $\frac{8}{3}$
			\item $\frac{32}{3}$
			\item $\frac{56}{3}$
			\item $\frac{14}{3}$
		\end{enumerate}
	\item The value of $\int_{0}^{\frac{\pi}{2}}\frac{\sin^{3} x}{\sin x + \cos x}dx$ is:

		\hfill{[JEE M 2019-9 April(M)]}
		\begin{enumerate}
			\item $\frac{\pi-2}{8}$
			\item $\frac{\pi-1}{4}$
			\item $\frac{\pi-2}{4}$
			\item $\frac{\pi-1}{2}$
		\end{enumerate}
	\item The area \brak{in sq. units} of the region $A=\{\brak{x,y}:x^{2}\leq y\leq x+2\}$ is:

		\hfill{[JEE M 2019-9 April(M)]}
		\begin{enumerate}
			\item $\frac{10}{3}$
			\item $\frac{9}{2}$
			\item $\frac{31}{6}$
			\item $\frac{13}{6}$
		\end{enumerate}
\end{enumerate}
\end{document}



